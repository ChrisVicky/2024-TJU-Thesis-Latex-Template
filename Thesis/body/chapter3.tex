% \addcontentsline{toc}{chapter}{结\quad 论} %添加到目录中
% \chapter*{结\quad 论}

\chapter{实验与分析}

\section{任务书}
任务书内容应包括原始依据、参考文献、设计内容和要求,其中原始依据要填写明确,原始依据不得少于200字,包括设计(论文)的工作基础、研究条件、应用环境、工作目的;设计(研究)内容和要求不得少于200字,包括设计(研究)内容、主要指标与技术参数,并根据课题性质对学生提出具体要求。

\section{开题报告}
开题报告要求不少于2000字,内容包括:课题的来源及意义,国内外发展状况,本课题的研究目标、研究内容、研究方法、研究手段和进度安排,实验方案的可行性分析和已具备的实验条件以及主要参考文献等。

\section{评阅书}

\subsection{指导教师评阅书}

指导教师评阅书中的“评阅意见” 不能少于200字,主要包括对开题报告、设计或研究内容、外文资料和译文、工作量、工作态度、设计或论文质量、创新性、应用性、论文写作、文本规范、存在的不足和综合评价等方面的评阅意见,应体现对所评阅论文的具体意见,要有针对性。

\subsection{评阅教师评阅书}

评阅教师评阅书中的“评阅意见”不能少于200字,主要包括对选题、设计或研究内容、外文资料和译文、工作量、设计或论文质量、创新性、应用性、论文写作、文本规范、存在的不足和综合评价等方面的评阅意见,应体现对所评阅论文的具体意见,要有针对性。

\section{答辩记录书}

答辩记录书中的“综合评价”不能少于100字,主要包括设计或研究内容、工作量、设计或论文质量和答辩情况;“答辩记录”主要包含答辩委员提出的问题和学生回答情况等,答辩记录由答辩小组秘书填写。

\section{外文资料和中文译文}

外文资料要与所做课题紧密联系,严禁抄袭有中文译本的外文资料,外文资料的选取要注明出处。可用 A4纸复印,如果打印,标题应采用不编号章标题样式,内容应采用正文样式。中文译文的字数一般为5000 ~ 6000汉字。

