% !Mode:: "TeX:UTF-8"

%%  可通过增加或减少 setup/format.tex中的
%%  第274行 \setlength{\@title@width}{8cm}中 8cm 这个参数来 控制封面中下划线的长度。

\cheading{天津大学~2016~届本科生毕业论文}      % 设置正文的页眉,需要填上对应的毕业年份
\ctitle{基于顾客有限理性预期的定价与供应链结构}    % 封面用论文标题,自己可手动断行
\caffil{管理与经济学部} % 学院名称
\csubject{工业工程}   % 专业名称
\cgrade{2012~级}            % 年级
\cauthor{秦昱博}            % 学生姓名
\cnumber{3012209017}        % 学生学号
\csupervisor{杨道箭}        % 导师姓名
\crank{副教授}              % 导师职称

\cdate{\the\year~年~\the\month~月~\the\day~日}

\cdeclaration{
本人声明:所呈交的毕业设计(论文),是本人在指导教师指导下,进行研究工作所取得的成果。除文中已经注明引用的内容外,本毕业设计(论文)中不包含任何他人已经发表或撰写过的研究成果。对本毕业设计(论文)所涉及的研究工作做出贡献的其他个人和集体,均已在论文中作了明确的说明。本毕业设计(论文)原创性声明的法律责任由本人承担。
}

\cabstract{
中文摘要一般为300~400字,简要介绍毕业设计(论文)的研究目的、方法、结果和结论,语言力求精炼。英文摘要应与中文摘要相对应。中英文摘要均要有关键词,一般为3~8个,中英文摘要要相互对应。

中文摘要。“摘要”两字之间空一个全角空格或两个半角空格,字体为宋体二号字加粗,居中显示,摘要内容采用正文样式。中文关键词与摘要内容间隔一行,无缩进左对齐书写。“关键词:”采用宋体四号字加粗,关键词内容采用正文样式,且换行不缩进。关键词之间用逗号分隔。

英文摘要。此部分皆为Times New Roman字体。“ABSTRACT”为二号字加粗,居中显示。英文摘要内容采用正文样式。英文关键词与英文摘要内容间隔一行,无缩进左对齐书写。“KEY WORDS:”为四号字加粗,英文关键词采用正文样式,且换行不缩进,关键词之间用逗号分隔,词义和中文关键词相同。“ABSTRACT”和“KEY WORDS”一律用大写字母,每个关键词的首字母要大写。
}

\ckeywords{关键词~1;关键词~2;关键词~3;……;关键词~7(关键词总共~3~—~7~个,最后一个关键词后面没有标点符号)}

\eabstract{
The upper bound of the number of Chinese characters is 400. The abstract aims at introducing the research purpose, research methods, research results, and research conclusion of graduation thesis, with refining words. Generally speaking, both the Chinese and English abstracts require the keywords, the number of which varies from 3 to 7, with a semicolon between adjacent words. The font of the English Abstract is Times New Roman, with the size of 12pt(small four).
}

\ekeywords{keyword 1, keyword 2, keyword 3, ……, keyword 7 (no punctuation at the end)}

\makecover

\clearpage
